\documentclass[specialist, subf, href, colorlinks=true, 12pt, times, mtpro, final]{disser}
\usepackage [russian] {babel}
\usepackage [utf8] {inputenc}
\usepackage {amsmath}
\usepackage {amsthm}
\usepackage {amssymb}
\usepackage{wrapfig}
\usepackage{enumitem}
\usepackage{amsfonts}
\usepackage{textcomp}
\usepackage{graphicx}
\usepackage{float}
\usepackage{caption}
\usepackage{algorithm}
\usepackage{xcolor}
\usepackage{hyperref}
\usepackage{pdfpages}
\usepackage{dsfont}

\theoremstyle{definition}
\newtheorem{defn}{Определение}[section]
\newtheorem{example}{Пример}[section]
\newtheorem{state}{Утверждение}[section]
\newtheorem{theorem}{Теорема}[section]

\definecolor{linkcolor}{HTML}{0000FF}
\definecolor{urlcolor}{HTML}{0000FF}
\definecolor{faded}{gray}{0.6}

\def\note{\textcolor{faded}}

\hypersetup{pdfstartview = FitH, linkcolor = linkcolor, urlcolor = urlcolor, colorlinks = true}

\begin{document}
    
    \tableofcontents

    \section*{Список вопросов}
    \note{В этом списке нужно расставить ссылки. Или оглавление сделать. Но так имхо удобнее, т.к. сюда можно по оглавлению возвращаться.}
    \begin{enumerate}
    \item \hyperref[1]{Свободная механическая система. Механическая система со связями. Аксиома освобождения от связей. Силы реакции связей. Уравнения связей. Пространство виртуальных скоростей. Принцип Даламбера – Лагранжа. Уравнения Лагранжа первого рода. Идеальные связи.}
    \item \hyperref[2]{Понятие об интегрируемости связей. Критерий интегрируемости (критерий Фробениуса, без доказательства). Обобщенные координаты системы с интегрируемыми связями. Уравнения дифференциальных связей в обобщенных координатах. Признак неинтегрируемости уравнений связей.}
    \item \hyperref[3]{Принцип Даламбера – Лагранжа в обобщенных координатах. Уравнения Лагранжа второго рода для голономных систем. Обобщенные силы. Работа сил на перемещении вдоль координатной линии. Случай потенциальных сил. Уравнения Лагранжа со множителями в обобщенных координатах.}
    \item \hyperref[4]{Энергия ускорений. Псевдоскорости. Уравнения Аппеля.}
    \item \hyperref[5]{Теоремы об изменении импульса, кинетического момента и кинетической энергии для систем со связями и следствия из них.}
    \item \hyperref[6]{Эквивалентность принципа Даламбера – Лагранжа и уравнений движения свободного твердого тела.}
    \item \hyperref[7]{Уравнения Лагранжа второго рода. Калибровка. Преобразование уравнений при замене координат.}
    \item \hyperref[8]{Первые интегралы уравнений Лагранжа для систем с потенциальными силами:
    интеграл Якоби, интеграл энергии, циклические интегралы. Поле симметрий. Теорема Нетер.}
    \item \hyperref[9]{Понижение порядка уравнений Лагранжа по Раусу. Функция Рауса. Приведенный потенциал. Уравнения Рауса.}
    \item \hyperref[10]{Задача Лагранжа о вращении тяжелого твердого тела вокруг неподвижной точки. Типичное движение оси динамической симметрии. Псевдорегулярная и регулярная прецессии. Способы реализации движений этих типов с заданным углом нутации.}
    \item \hyperref[11]{Равновесия системы уравнений Лагранжа. Соответствующие движения механической системы. Уравнения равновесия. Устойчивость равновесий по Ляпунову. Теорема Лагранжа – Дирихле о достаточном условии устойчивости равновесия.}
    \item \hyperref[12]{Линеаризация уравнений Лагранжа в окрестности состояния равновесия. Существование нормальных координат. Линеаризованные уравнения в нормальных координатах, их интегрирование. Уравнение для собственных чисел. Независимость собственных чисел от выбора координат. Собственные векторы. Выражение матрицы преобразования к нормальным координатам через компоненты собственных векторов.}
    \item \hyperref[13]{Преобразование уравнений Лагранжа в уравнения Гамильтона. Обобщенные импульсы. Явный вид функции Гамильтона для уравнений механики.}
    \item \hyperref[14]{Первые интегралы уравнений Гамильтона. Скобки Пуассона в канонических координатах. Свойства скобок Пуассона, тождество Якоби. Простейшие первые интегралы в случаях независимости функции Гамильтона от времени, наличия циклических координат, отделения переменных.}
    \item \hyperref[15]{Аналог теоремы Нетер для уравнений Гамильтона. Теорема о скобке Пуассона двух первых интегралов. Скобки Пуассона компонент кинетического момента свободной точки.}
    \item \hyperref[16]{Канонические преобразования. Определение, его переформулировка в терминах скобок Пуассона, интерпретация в случае системы с одной степенью свободы.
    Примеры: тождественное, обратное к каноническому, ортогональное преобразование фазовой плоскости, полярные канонические координаты, каноническая перестановка, каноническое изменение масштабов на координатных осях.}
    \item \hyperref[17]{Критерии каноничности преобразования.}
    \item \hyperref[18]{Преобразование уравнений Гамильтона при канонических преобразованиях.}
    \item \hyperref[19]{Производящие функции канонических преобразований (свободного и стандартного). Выражение новой функции Гамильтона через производящие функции в этих случаях.
    Производящие функции для преобразований: тождественного, перехода к полярным каноническим координатам, канонической перестановки, канонического изменения масштабов на координатных осях.}
    \item \hyperref[20]{Уравнения для производящих функций целенаправленных канонических преобразований. Уравнение Гамильтона – Якоби, его полный интеграл. Теорема Якоби. Нахождение полного интеграла в случаях независимости функции Гамильтона от времени, наличия циклических координат, отделения переменных.}
    \item \hyperref[21]{Теорема Лиувилля об интегрировании в квадратурах системы уравнений Гамильтона.}
    \item \hyperref[22]{Канонические отображения в фазовом пространстве в канонических координатах. Каноничность отображения вдоль решений системы уравнений Гамильтона. Сохранение фазового объема при этом отображении. Интегральный инвариант Пуанкаре.}
    \item \hyperref[23]{Интегральный инвариант Пуанкаре – Картана.}
    \item \hyperref[24]{Вариационный принцип Гамильтона в фазовом пространстве системы уравнений Гамильтона.}
    \item \hyperref[25]{Вариационный принцип Гамильтона на конфигурационном многообразии системы уравнений Лагранжа.}
    \item \hyperref[26]{Метод усреднения для систем Гамильтона в стандартной форме. Математический маятник с точкой подвеса, движущейся по эллипсу.}
    \end{enumerate}

    \section{Свободная механическая система. Механическая система со связями. Аксиома освобождения от связей. Силы реакции связей. Уравнения связей. Пространство виртуальных скоростей. Принцип Даламбера – Лагранжа. Уравнения Лагранжа первого рода. Идеальные связи.}
    \label{1}
     \hyperlink {first_lects.1}{Лекции} \\
    
    \section{Понятие об интегрируемости связей. Критерий интегрируемости (критерий Фробениуса, без доказательства). Обобщенные координаты системы с интегрируемыми связями. Уравнения дифференциальных связей в обобщенных координатах. Признак неинтегрируемости уравнений связей.}
    \label{2}
	\hyperlink {first_lects.7}{Лекции} \\

	\section{Принцип Даламбера – Лагранжа в обобщенных координатах. Уравнения Лагранжа второго рода для голономных систем. Обобщенные силы. Работа сил на перемещении вдоль координатной линии. Случай потенциальных сил. Уравнения Лагранжа со множителями в обобщенных координатах.}
	 \label{3}
	\hyperlink {first_lects.14}{Лекции} \\
	
    \section{Энергия ускорений. Псевдоскорости. Уравнения Аппеля.}
     \label{4}
   	\hyperlink {first_lects.19}{Лекции} \\
   	
    \section{Теоремы об изменении импульса, кинетического момента и кинетической энергии для систем со связями и следствия из них.}
    \label{5}
    \hyperlink {first_lects.22}{Лекции} \\
    
    \section{Эквивалентность принципа Даламбера – Лагранжа и уравнений движения свободного твердого тела.}
     \label{6}
    \hyperlink {first_lects.27}{Лекции} \\
    
    \section{Уравнения Лагранжа второго рода. Калибровка. Преобразование уравнений при замене координат.}
     \label{7}
    \hyperlink {first_lects.29}{Лекции} \\
    
    \section{Первые интегралы уравнений Лагранжа для систем с потенциальными силами:    интеграл Якоби, интеграл энергии, циклические интегралы. Поле симметрий. Теорема Нетер.}
     \label{8}
    \hyperlink {lects.1}{Лекции} \\
    
    \section{Понижение порядка уравнений Лагранжа по Раусу. Функция Рауса. Приведенный потенциал. Уравнения Рауса.}
     \label{9}
    \hyperlink {lects.3}{Лекции} \\
    
    \section{Задача Лагранжа о вращении тяжелого твердого тела вокруг неподвижной точки. Типичное движение оси динамической симметрии. Псевдорегулярная и регулярная прецессии. Способы реализации движений этих типов с заданным углом нутации.}
     \label{10}
    \hyperlink {lects.6}{Лекции} \\
    
    \section{Равновесия системы уравнений Лагранжа. Соответствующие движения механической системы. Уравнения равновесия. Устойчивость равновесий по Ляпунову. Теорема Лагранжа – Дирихле о достаточном условии устойчивости равновесия.}
     \label{11}
    \hyperlink {lects.10}{Лекции} \\
    
    \section{Линеаризация уравнений Лагранжа в окрестности состояния равновесия. Существование нормальных координат. Линеаризованные уравнения в нормальных координатах, их интегрирование. Уравнение для собственных чисел. Независимость собственных чисел от выбора координат. Собственные векторы. Выражение матрицы преобразования к нормальным координатам через компоненты собственных векторов.}
     \label{12}
    \hyperlink {lects.11}{Лекции (конец страницы)} \\
    
    \section{Преобразование уравнений Лагранжа в уравнения Гамильтона. Обобщенные импульсы. Явный вид функции Гамильтона для уравнений механики.}
     \label{13}
     \hyperlink {lects.16}{Лекции} \\
    
        
    \section{Первые интегралы уравнений Гамильтона. Скобки Пуассона в канонических координатах. Свойства скобок Пуассона, тождество Якоби. Простейшие первые интегралы в случаях независимости функции Гамильтона от времени, наличия циклических координат, отделения переменных.}
     \label{14}
    \hyperlink {lects.18}{Лекции} \\
    
    \section{Аналог теоремы Нетер для уравнений Гамильтона. Теорема о скобке Пуассона двух первых интегралов. Скобки Пуассона компонент кинетического момента свободной точки.}
     \label{15}
    \hyperlink {lects.18}{Лекции} \\
    
    \section{Канонические преобразования. Определение, его переформулировка в терминах скобок Пуассона, интерпретация в случае системы с одной степенью свободы. Примеры: тождественное, обратное к каноническому, ортогональное преобразование фазовой плоскости, полярные канонические координаты, каноническая перестановка, каноническое изменение масштабов на координатных осях.}
     \label{16}
    \hyperlink {lects.23}{Лекции} \\
    
    \section{Критерии каноничности преобразования.}
     \label{17}
    \hyperlink {lects.25}{Лекции} \\
    
    \section{Преобразование уравнений Гамильтона при канонических преобразованиях.}
     \label{18}
    \hyperlink {lects.27}{Лекции} \\
    
    \section{Производящие функции канонических преобразований (свободного и стандартного). Выражение новой функции Гамильтона через производящие функции в этих случаях. Производящие функции для преобразований: тождественного, перехода к полярным каноническим координатам, канонической перестановки, канонического изменения масштабов на координатных осях.}
     \label{19}
    \hyperlink {lects.28}{Лекции (низ страницы)} \\
    
    \section{Уравнения для производящих функций целенаправленных канонических преобразований. Уравнение Гамильтона – Якоби, его полный интеграл. Теорема Якоби. Нахождение полного интеграла в случаях независимости функции Гамильтона от времени, наличия циклических координат, отделения переменных.}
     \label{20}
    \hyperlink {lects.31}{Лекции} \\
    
    \section{Теорема Лиувилля об интегрировании в квадратурах системы уравнений Гамильтона.}
     \label{21}
    \hyperlink {lects.37}{Лекции} \\
    
    \section{Канонические отображения в фазовом пространстве в канонических координатах. Каноничность отображения вдоль решений системы уравнений Гамильтона. Сохранение фазового объема при этом отображении. Интегральный инвариант Пуанкаре.}
     \label{22}
	\hyperlink {lects.39}{Лекции} \\
    
    \section{Интегральный инвариант Пуанкаре – Картана.}
     \label{23}
    \hyperlink {lects.42}{Лекции} \\
    
    \section{Вариационный принцип Гамильтона в фазовом пространстве системы уравнений Гамильтона.}
     \label{24}
    \hyperlink {lects.44}{Лекции} \\
    
    \section{Вариационный принцип Гамильтона на конфигурационном многообразии системы уравнений Лагранжа.}
     \label{25}
    \hyperlink {lects.45}{Лекции} \\
    
    \section{Метод усреднения для систем Гамильтона в стандартной форме. Математический маятник с точкой подвеса, движущейся по эллипсу.}
     \label{26}
    \hyperlink {lects.47}{Лекции (сразу над формулами)} \\
    
    \includepdf[pages=-, link, linkname = first_lects]{first_lections.pdf}
    \includepdf[pages=-, link, linkname = lects]{Mekhanika_Dist.pdf}
\end{document}
