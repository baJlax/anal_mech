\documentclass[specialist, subf, href, colorlinks=true, 12pt, times, mtpro, final]{disser}
\usepackage [russian] {babel}
\usepackage [utf8] {inputenc}
\usepackage {amsmath}
\usepackage {amsthm}
\usepackage {amssymb}
\usepackage{wrapfig}
\usepackage{enumitem}
\usepackage{amsfonts}
\usepackage{textcomp}
\usepackage{graphicx}
\usepackage{float}
\usepackage{caption}
\usepackage{algorithm}
\usepackage{xcolor}
\usepackage{hyperref}
\usepackage{pdfpages}
\usepackage{dsfont}

\theoremstyle{definition}
\newtheorem{defn}{Определение}[section]
\newtheorem{example}{Пример}[section]
\newtheorem{state}{Утверждение}[section]
\newtheorem{theorem}{Теорема}[section]

\definecolor{linkcolor}{HTML}{0000FF}
\definecolor{urlcolor}{HTML}{0000FF}
\definecolor{faded}{gray}{0.6}

\def\note{\textcolor{faded}}

\hypersetup{pdfstartview = FitH, linkcolor = linkcolor, urlcolor = urlcolor, colorlinks = true}

\begin{document}
    
    \tableofcontents

    \section*{Список вопросов}
    \note{В этом списке нужно расставить ссылки. Или оглавление сделать. Но так имхо удобнее, т.к. сюда можно по оглавлению возвращаться.}
    \begin{enumerate}
    \item Свободная механическая система. Механическая система со связями. Аксиома освобождения от связей. Силы реакции связей. Уравнения связей. Пространство виртуальных скоростей. Принцип Даламбера – Лагранжа. Уравнения Лагранжа первого рода. Идеальные связи.
    \item Понятие об интегрируемости связей. Критерий интегрируемости (критерий Фробениуса, без доказательства). Обобщенные координаты системы с интегрируемыми связями. Уравнения дифференциальных связей в обобщенных координатах. Признак неинтегрируемости уравнений связей.
    \item Принцип Даламбера – Лагранжа в обобщенных координатах. Уравнения Лагранжа второго рода для голономных систем. Обобщенные силы. Работа сил на перемещении вдоль координатной линии. Случай потенциальных сил. Уравнения Лагранжа со множителями в обобщенных координатах.
    \item Энергия ускорений. Псевдоскорости. Уравнения Аппеля.
    \item Теоремы об изменении импульса, кинетического момента и кинетической энергии для систем со связями и следствия из них.
    \item Эквивалентность принципа Даламбера – Лагранжа и уравнений движения свободного твердого тела.
    \item Уравнения Лагранжа второго рода. Калибровка. Преобразование уравнений при замене координат.
    \item Первые интегралы уравнений Лагранжа для систем с потенциальными силами:
    интеграл Якоби, интеграл энергии, циклические интегралы. Поле симметрий. Теорема Нетер.
    \item Понижение порядка уравнений Лагранжа по Раусу. Функция Рауса. Приведенный потенциал. Уравнения Рауса.
    \item Задача Лагранжа о вращении тяжелого твердого тела вокруг неподвижной точки. Типичное движение оси динамической симметрии. Псевдорегулярная и регулярная прецессии. Способы реализации движений этих типов с заданным углом нутации.
    \item Равновесия системы уравнений Лагранжа. Соответствующие движения механической системы. Уравнения равновесия. Устойчивость равновесий по Ляпунову. Теорема Лагранжа – Дирихле о достаточном условии устойчивости равновесия.
    \item Линеаризация уравнений Лагранжа в окрестности состояния равновесия. Существование нормальных координат. Линеаризованные уравнения в нормальных координатах, их интегрирование. Уравнение для собственных чисел. Независимость собственных чисел от выбора координат. Собственные векторы. Выражение матрицы преобразования к нормальным координатам через компоненты собственных векторов.
    \item Преобразование уравнений Лагранжа в уравнения Гамильтона. Обобщенные импульсы. Явный вид функции Гамильтона для уравнений механики.
    \item Первые интегралы уравнений Гамильтона. Скобки Пуассона в канонических координатах. Свойства скобок Пуассона, тождество Якоби. Простейшие первые интегралы в случаях независимости функции Гамильтона от времени, наличия циклических координат, отделения переменных.
    \item Аналог теоремы Нетер для уравнений Гамильтона. Теорема о скобке Пуассона двух первых интегралов. Скобки Пуассона компонент кинетического момента свободной точки.
    \item Канонические преобразования. Определение, его переформулировка в терминах скобок Пуассона, интерпретация в случае системы с одной степенью свободы.
    Примеры: тождественное, обратное к каноническому, ортогональное преобразование фазовой плоскости, полярные канонические координаты, каноническая перестановка, каноническое изменение масштабов на координатных осях.
    \item Критерии каноничности преобразования.
    \item Преобразование уравнений Гамильтона при канонических преобразованиях.
    \item Производящие функции канонических преобразований (свободного и стандартного). Выражение новой функции Гамильтона через производящие функции в этих случаях.
    Производящие функции для преобразований: тождественного, перехода к полярным каноническим координатам, канонической перестановки, канонического изменения масштабов на координатных осях.
    \item Уравнения для производящих функций целенаправленных канонических преобразований. Уравнение Гамильтона – Якоби, его полный интеграл. Теорема Якоби. Нахождение полного интеграла в случаях независимости функции Гамильтона от времени, наличия циклических координат, отделения переменных.
    \item Теорема Лиувилля об интегрировании в квадратурах системы уравнений Гамильтона.
    \item Канонические отображения в фазовом пространстве в канонических координатах. Каноничность отображения вдоль решений системы уравнений Гамильтона. Сохранение фазового объема при этом отображении. Интегральный инвариант Пуанкаре.
    \item Интегральный инвариант Пуанкаре – Картана.
    \item Вариационный принцип Гамильтона в фазовом пространстве системы уравнений Гамильтона.
    \item Вариационный принцип Гамильтона на конфигурационном многообразии системы уравнений Лагранжа.
    \item Метод усреднения для систем Гамильтона в стандартной форме. Математический маятник с точкой подвеса, движущейся по эллипсу.
    \end{enumerate}

    \section{Свободная механическая система. Механическая система со связями. Аксиома освобождения от связей. Силы реакции связей. Уравнения связей. Пространство виртуальных скоростей. Принцип Даламбера – Лагранжа. Уравнения Лагранжа первого рода. Идеальные связи.}
    aaa
    
    \section{Понятие об интегрируемости связей. Критерий интегрируемости (критерий Фробениуса, без доказательства). Обобщенные координаты системы с интегрируемыми связями. Уравнения дифференциальных связей в обобщенных координатах. Признак неинтегрируемости уравнений связей.}
	aaa

	\section{Принцип Даламбера – Лагранжа в обобщенных координатах. Уравнения Лагранжа второго рода для голономных систем. Обобщенные силы. Работа сил на перемещении вдоль координатной линии. Случай потенциальных сил. Уравнения Лагранжа со множителями в обобщенных координатах.}
	aaa
	
    \section{Энергия ускорений. Псевдоскорости. Уравнения Аппеля.}
   	aaa
   	
    \section{Теоремы об изменении импульса, кинетического момента и кинетической энергии для систем со связями и следствия из них.}
    aaa
    
    \section{Эквивалентность принципа Даламбера – Лагранжа и уравнений движения свободного твердого тела.}
    aaa
    
    \section{Уравнения Лагранжа второго рода. Калибровка. Преобразование уравнений при замене координат.}
    aaa
    
    \section{Первые интегралы уравнений Лагранжа для систем с потенциальными силами:    интеграл Якоби, интеграл энергии, циклические интегралы. Поле симметрий. Теорема Нетер.}
    aaa
    
    \section{Понижение порядка уравнений Лагранжа по Раусу. Функция Рауса. Приведенный потенциал. Уравнения Рауса.}
    aaa
    
    \section{Задача Лагранжа о вращении тяжелого твердого тела вокруг неподвижной точки. Типичное движение оси динамической симметрии. Псевдорегулярная и регулярная прецессии. Способы реализации движений этих типов с заданным углом нутации.}
    aaa
    
    \section{Равновесия системы уравнений Лагранжа. Соответствующие движения механической системы. Уравнения равновесия. Устойчивость равновесий по Ляпунову. Теорема Лагранжа – Дирихле о достаточном условии устойчивости равновесия.}
    aaa
    
    \section{Линеаризация уравнений Лагранжа в окрестности состояния равновесия. Существование нормальных координат. Линеаризованные уравнения в нормальных координатах, их интегрирование. Уравнение для собственных чисел. Независимость собственных чисел от выбора координат. Собственные векторы. Выражение матрицы преобразования к нормальным координатам через компоненты собственных векторов.}
    aaa
    
    \section{Преобразование уравнений Лагранжа в уравнения Гамильтона. Обобщенные импульсы. Явный вид функции Гамильтона для уравнений механики.}
    aaa
    
    \section{Первые интегралы уравнений Гамильтона. Скобки Пуассона в канонических координатах. Свойства скобок Пуассона, тождество Якоби. Простейшие первые интегралы в случаях независимости функции Гамильтона от времени, наличия циклических координат, отделения переменных.}
    aaa
    
    \section{Аналог теоремы Нетер для уравнений Гамильтона. Теорема о скобке Пуассона двух первых интегралов. Скобки Пуассона компонент кинетического момента свободной точки.}
    aaa
    
    \section{Канонические преобразования. Определение, его переформулировка в терминах скобок Пуассона, интерпретация в случае системы с одной степенью свободы. Примеры: тождественное, обратное к каноническому, ортогональное преобразование фазовой плоскости, полярные канонические координаты, каноническая перестановка, каноническое изменение масштабов на координатных осях.}
    aaa
    
    \section{Критерии каноничности преобразования.}
    aaa
    
    \section{Преобразование уравнений Гамильтона при канонических преобразованиях.}
    aaa
    
    \section{Производящие функции канонических преобразований (свободного и стандартного). Выражение новой функции Гамильтона через производящие функции в этих случаях. Производящие функции для преобразований: тождественного, перехода к полярным каноническим координатам, канонической перестановки, канонического изменения масштабов на координатных осях.}
    aaa
    
    \section{Уравнения для производящих функций целенаправленных канонических преобразований. Уравнение Гамильтона – Якоби, его полный интеграл. Теорема Якоби. Нахождение полного интеграла в случаях независимости функции Гамильтона от времени, наличия циклических координат, отделения переменных.}
    aaa
    
    \section{Теорема Лиувилля об интегрировании в квадратурах системы уравнений Гамильтона.}
    aaa
    
    \section{Канонические отображения в фазовом пространстве в канонических координатах. Каноничность отображения вдоль решений системы уравнений Гамильтона. Сохранение фазового объема при этом отображении. Интегральный инвариант Пуанкаре.}
	aaa
    
    \section{Интегральный инвариант Пуанкаре – Картана.}
    aaa
    
    \section{Вариационный принцип Гамильтона в фазовом пространстве системы уравнений Гамильтона.}
    aaa
    
    \section{Вариационный принцип Гамильтона на конфигурационном многообразии системы уравнений Лагранжа.}
    aaa
    
    \section{Метод усреднения для систем Гамильтона в стандартной форме. Математический маятник с точкой подвеса, движущейся по эллипсу.}
    aaa
    
    \includepdf[pages=-, link, linkname = lects]{Mekhanika_Dist.pdf}
\end{document}
